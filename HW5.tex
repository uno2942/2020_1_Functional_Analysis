%Calculus Homework
\documentclass[a4paper, 12pt]{article}

%================================================================================
%Package
	\usepackage{amsmath, amsthm, amssymb, latexsym, mathtools, mathrsfs, physics}
	\usepackage{dsfont, txfonts, soul, stackrel, tikz-cd, graphicx, titlesec, etoolbox}
	\DeclareGraphicsExtensions{.pdf,.png,.jpg}
	\usepackage{fancyhdr}
	\usepackage[shortlabels]{enumitem}
	\usepackage[pdfmenubar=true, pdfborder	={0 0 0 [3 3]}]{hyperref}
	\usepackage{kotex}

%================================================================================
\usepackage{verbatim}
\usepackage{physics}
\usepackage{makebox}
\usepackage{pst-node, auto-pst-pdf}

%================================================================================
%Layout
	%Page layout
	\addtolength{\hoffset}{-50pt}
	\addtolength{\headheight}{+10pt}
	\addtolength{\textwidth}{+75pt}
	\addtolength{\voffset}{-50pt}
	\addtolength{\textheight}{+75pt}
	\newcommand{\Space}{1em}
	\newcommand{\Vspace}{\vspace{\Space}}
	\newcommand{\ran}{\textrm{ran }}
	\setenumerate{listparindent=\parindent}

%================================================================================
%Statement
	\newtheoremstyle{Mytheorem}%
	{1em}{1em}%
	{\slshape}{}%
	{\bfseries}{.}%
	{ }{}

	\newtheoremstyle{Mydefinition}%
	{1em}{1em}%
	{}{}%
	{\bfseries}{.}%
	{ }{}

	\theoremstyle{Mydefinition}
	\newtheorem{statement}{Statement}
	\newtheorem{definition}[statement]{Definition}
	\newtheorem{definitions}[statement]{Definitions}
	\newtheorem{remark}[statement]{Remark}
	\newtheorem{remarks}[statement]{Remarks}
	\newtheorem{example}[statement]{Example}
	\newtheorem{examples}[statement]{Examples}
	\newtheorem{question}[statement]{Question}
	\newtheorem{questions}[statement]{Questions}
	\newtheorem{problem}[statement]{Problem}
	\newtheorem{exercise}{Exercise}[section]
	\newtheorem*{comment*}{Comment}
	%\newtheorem{exercise}{Exercise}[subsection]

	\theoremstyle{Mytheorem}
	\newtheorem{theorem}[statement]{Theorem}
	\newtheorem{corollary}[statement]{Corollary}
	\newtheorem{corollaries}[statement]{Corollaries}
	\newtheorem{proposition}[statement]{Proposition}
	\newtheorem{lemma}[statement]{Lemma}
	\newtheorem{claim}{Claim}
	\newtheorem{claimproof}{Proof of claim}[claim]

%================================================================================
%Header & footer
	\fancypagestyle{myfency}{%Plain
	\fancyhf{}
	\fancyhead[L]{}
	\fancyhead[C]{}
	\fancyhead[R]{}
	\fancyfoot[L]{}
	\fancyfoot[C]{}
	\fancyfoot[R]{\thepage}
	\renewcommand{\headrulewidth}{0.4pt}
	\renewcommand{\footrulewidth}{0pt}}

	\fancypagestyle{myfirstpage}{%Firstpage
	\fancyhf{}
	\fancyhead[L]{}
	\fancyhead[C]{}
	\fancyhead[R]{}
	\fancyfoot[L]{}
	\fancyfoot[C]{}
	\fancyfoot[R]{\thepage}
	\renewcommand{\headrulewidth}{0pt}
	\renewcommand{\footrulewidth}{0pt}}

	\pagestyle{myfency}

%================================================================================

%***************************
%*** Additional Command ****
%***************************

\DeclareMathOperator{\cl}{cl}

%================================================================================
%Document
\begin{document}
\thispagestyle{myfirstpage}
\begin{center}
	\Large{Functional Analysis HW3}
\end{center}
박성빈, 수학과, 20202120

\noindent \textbf{2.15}

\begin{proof}[Solution]
First, I'll prove lemmas.
\begin{lemma}
If $A\subset \mathscr{H}$, then $A^\perp = (\cl A)^\perp$.
\end{lemma}

\begin{proof}
For $A\subset B$, $A^\perp\supset B^\perp$ by definition, so $A^\perp \supset (\cl A)^\perp$. Conversely, if $x\in A^\perp$, for any $c\in\cl A$ with $\{a_i\}\subset A$ such that $a_i\rightarrow c$, $0 = \langle x, a_i\rangle\rightarrow \langle x, c\rangle$, so $x\in (\cl A)^\perp$.
\end{proof}

\begin{lemma}
If $A \subset \mathscr{H}$ is a linear subspace, then $(A^\perp)^\perp = \cl A$.
\end{lemma}

\begin{proof}
In the previous homework, we showed that $\vee A$ is the closure of $\{\sum_{i=1}^n \alpha_i : n\geq 1, \alpha_i f_i\in \varmathbb{F}, f_i\in A\}$. Since $A$ is already a linear space, $\vee A = \cl A$. Also, $A^\perp = (\cl(A))^\perp$ by lemma 1. Therefore, $(A^\perp)^\perp = ((\cl A)^\perp)^\perp = \cl A$. 
\end{proof}

\begin{lemma}
For $T\in \mathscr{B}(\mathscr{H})$, $\cl(\ran T) = (\ker T^*)^\perp$ and $\cl(\ran T^*) = (\ker T)^\perp$.
\end{lemma}

\begin{proof}
Since $\ran T$ and $\ran T^*$ are already linear subspaces, by lemma 1, 2, $\cl(\ran T) = ((\ran T)^\perp)^\perp = (\ker T^*)^\perp$ and $\cl(\ran T^*) = (\ker T)^\perp$.
\end{proof}

($\Leftarrow$) Since $\ran A$ forms a dense subset in $\mathscr{H}$, $\cl(\ran A) = \mathscr{H}$ and $\ker A^* = 0$. Now, let $x\in \ker A$. Since $A$ is normal, $AA^* x = A^* A x = 0$, so $A^* x\in \ker A$. It implies that $A^*x \in \ker A \cap \ran A^*$. As $\ran A^* \subset \cl(\ran A^*) = (\ker A)^\perp$, $\ker A \cap \ran A^* = 0$ and $x = 0$, so $\ker A = 0$.

($\Rightarrow$) Assume $\cl(\ran A)\neq \mathscr{H}$, and choose a nonzero $x$ in $(\cl(\ran A))^\perp = (\ran A)^\perp = \ker A^*$, then $A^* x = 0$ and $A^* A x = 0$. Therefore, $Ax\in \ran A \cap \ker A^* = 0$. As $\ran A\subset (\ker A^*)^\perp$, $\ker A\neq 0$, which is contradiction to the assumption that $A$ is not injective. Therefore, $\cl(\ran A) = \mathscr{H}$.

Consider the unilateral shift operator $S:l^2\rightarrow l^2$ by $S^*(\alpha_1, \alpha_2, \ldots) = (0, \alpha_1, \alpha_2, \ldots)$. Then $\ker S = 0$, but the range of $S$ is not dense in $l^2$ since first coordinate is always $0$ in $\ran S$.

Consider the backward shift operator, i.e., $S^*:l^2\rightarrow l^2$ defined by $S^*(\alpha_1, \alpha_2, \ldots) = (\alpha_2, \alpha_3, \ldots)$. It is surjective, but $(1,0,0,0,\ldots)\in \ker S^*$.
\end{proof}



\noindent \textbf{3.10}

\begin{proof}[Solution]
Set $\mathscr{H} = l^2$ and $e_i$ be the element in $l^2$ such that only $i$th coordinate is $1$. Let $\pi_i$ be the projection to $i$th coordinate, i.e. $\pi_i(x) = \langle x, e_i\rangle$. Let $M$ be the closed linear span of $\{a_n = \frac{1}{\sqrt{2n-1}}e_{2n-1}+\frac{1}{\sqrt{2n}}e_{2n}:n\in\varmathbb{N}\}$ and $N$ be the closed linear span of $\{b_n = \frac{1}{\sqrt{2n}}e_{2n}+\frac{1}{\sqrt{2n+1}}e_{2n + 1}:n\in\varmathbb{N}\}$.

Assume $x\in M\cap N$, Since $\pi_1(b) = 0$ for all $b\in N$, $\pi_1(x) = 0$. Since the first coordinate is joined with second coordinate in $M$, $\pi_2(x)=0$, and it applies to third coordinate in $N$. Repeating this process, we get $\pi_i(x)=0$ for all $i$ by induction, so $x = 0$.

I'll show that $e_i$ can be approximated by $M+N$ for any $i$. If $i = 2s-1$ is an odd number, then
\begin{equation}
    e_i -\frac{\sqrt{2s-1}}{\sqrt{2t+1}}e_{2t+1}= \sqrt{2s-1}((a_s -b_s) + \ldots + (a_t - b_t)) = \sqrt{2s-1}(\sum_{i=s}^t a_i - \sum_{i=s}^t b_i)
\end{equation}
and $\norm{\left(e_i - \frac{\sqrt{2s-1}}{\sqrt{2t+1}}e_{2t+1}\right) - e_i} = \frac{2s-1}{2t+1}$ for $t>s$. This argument can be applied when $i$ is an even number, so it shows that $e_i\in \cl(M+N)$ for any $i$.

I'll show that $e_1\notin M+N$. Assume $e_1 = a+b$ for some $a\in M$, $b\in N$. Since $M, N$ are again Hilbert spaces, by setting $\mathscr{E}_M = \{a_n/\norm{a_n}\}$, $\mathscr{E}_N = \{b_n/\norm{b_n}\}$, we get $\vee \mathscr{E}_M = M$ and $\vee \mathscr{E}_N = N$ by construction. Therefore, $a = \sum_{i=1}^\infty \alpha_i a_i$ and $b = \sum_{i=1}^\infty \beta_i b_i$ for $\alpha_i,\beta_i\in \varmathbb{F}$. To make the first coordinate $1$, $\alpha_1$ should be $1$, and to compensate second coordinate, $\beta_1$ should be $-1$. Repeating this process, we get $a = \sum_{i=1}^\infty a_i$ and $b = \sum_{i=1}^\infty -b_i$, and both are not in $l^2$. Therefore, the assumption is false and $M+N\neq \mathscr{H}$.
\end{proof}


\noindent \textbf{3.12}

\begin{proof}[Solution]
Take $B = \{z\in\varmathbb{C}: \abs{z-1/2}\leq 1/8\}$. Let $M = \{f\in L^2(\mu): f(x) = 0 \textrm{  on }x\in B^c \textrm{ a.e.}\}$ and $N = \{f\in L^2(\mu): f(x) = 0 \textrm{  on }x\in B \textrm{ a.e.}\}$. Both are closed linear subspaces of $\mathscr{H}$. For example, if $f_i\rightarrow f$ and $\{f_i\}\subset M$, but $f\notin M$, then $\int_{\varmathbb{D}} \abs{f_i-f}^2 d\mu \geq \int_{B^c} \abs{f}^2 d\mu > 0$. 

I'll first show that $M^\perp = N$. For $f\in M$, $g\in N$ $\int_{\varmathbb{D}} f\overline{g} d\mu = 0$, so $N\subset M^\perp$. Conversely, if there exists $g\in M^\perp$ such that $g\neq 0$ on $E\subset B$ and $\mu(E)>0$, then take $f = g\chi_E$ in $M$, then $\int_{\varmathbb{D}} f \overline{g} d\mu = \int_{E} g \overline{g} d\mu > 0$, which is contradiction. Therefore, $M^\perp \subset N$.

Since multiplying $z$ does not change support of a function, $AM\subset M$ and $AN = AM^\perp \subset M^\perp$.

To make a counterexample, set $L=\{1,z, z^2, \ldots\}$ and $M = \vee L$. Since $\int_{\varmathbb{D}} z^n \overline{(\overline{z})} d\mu = 0$ for any $n\in \varmathbb{N}\cup \{0\}$, $\overline{z}\in M^\perp$. It is easy to see that $AM\subset M$ as for any $m\in M$ with a sequence of polynomial functions $p_i(z)$ such that $p_i(z)\rightarrow m$ in $L^2$ norm, $$\int_{\varmathbb{D}} \abs{zm-zp_i(z)}^2 d\mu \leq \norm{z}^2_{L^\infty(\varmathbb{D})} \int_{\varmathbb{D}} \abs{m-p_i(z)}^2 d\mu$$, so $zp_i(z)\rightarrow zm$ and $Am\in M$. However, $A\overline{z} = z\overline{z}\notin M^\perp$ since $\int_{\varmathbb{D}} (z\overline{z})(1) d\mu = \int_{\varmathbb{D}} \abs{z}^2 d\mu> 0$.
\end{proof}

%________________________________________________________________________
\end{document}

%================================================================================