%Calculus Homework
\documentclass[a4paper, 12pt]{article}

%================================================================================
%Package
	\usepackage{amsmath, amsthm, amssymb, latexsym, mathtools, mathrsfs, physics}
	\usepackage{dsfont, txfonts, soul, stackrel, tikz-cd, graphicx, titlesec, etoolbox}
	\DeclareGraphicsExtensions{.pdf,.png,.jpg}
	\usepackage{fancyhdr}
	\usepackage[shortlabels]{enumitem}
	\usepackage[pdfmenubar=true, pdfborder	={0 0 0 [3 3]}]{hyperref}
	\usepackage{kotex}

%================================================================================
\usepackage{verbatim}
\usepackage{physics}
\usepackage{makebox}
\usepackage{pst-node, auto-pst-pdf}

%================================================================================
%Layout
	%Page layout
	\addtolength{\hoffset}{-50pt}
	\addtolength{\headheight}{+10pt}
	\addtolength{\textwidth}{+75pt}
	\addtolength{\voffset}{-50pt}
	\addtolength{\textheight}{+75pt}
	\newcommand{\Space}{1em}
	\newcommand{\Vspace}{\vspace{\Space}}
	\newcommand{\ran}{\textrm{ran }}
	\newcommand{\gra}{\textrm{gra }}
	\setenumerate{listparindent=\parindent}

%================================================================================
%Statement
	\newtheoremstyle{Mytheorem}%
	{1em}{1em}%
	{\slshape}{}%
	{\bfseries}{.}%
	{ }{}

	\newtheoremstyle{Mydefinition}%
	{1em}{1em}%
	{}{}%
	{\bfseries}{.}%
	{ }{}

	\theoremstyle{Mydefinition}
	\newtheorem{statement}{Statement}
	\newtheorem{definition}[statement]{Definition}
	\newtheorem{definitions}[statement]{Definitions}
	\newtheorem{remark}[statement]{Remark}
	\newtheorem{remarks}[statement]{Remarks}
	\newtheorem{example}[statement]{Example}
	\newtheorem{examples}[statement]{Examples}
	\newtheorem{question}[statement]{Question}
	\newtheorem{questions}[statement]{Questions}
	\newtheorem{problem}[statement]{Problem}
	\newtheorem{exercise}{Exercise}[section]
	\newtheorem*{comment*}{Comment}
	%\newtheorem{exercise}{Exercise}[subsection]

	\theoremstyle{Mytheorem}
	\newtheorem{theorem}[statement]{Theorem}
	\newtheorem{corollary}[statement]{Corollary}
	\newtheorem{corollaries}[statement]{Corollaries}
	\newtheorem{proposition}[statement]{Proposition}
	\newtheorem{lemma}[statement]{Lemma}
	\newtheorem{claim}{Claim}
	\newtheorem{claimproof}{Proof of claim}[claim]
	\newenvironment{myproof1}[1][\proofname]{%
  \proof[\textit Proof of problem #1]%
}{\endproof}

%================================================================================
%Header & footer
	\fancypagestyle{myfency}{%Plain
	\fancyhf{}
	\fancyhead[L]{}
	\fancyhead[C]{}
	\fancyhead[R]{}
	\fancyfoot[L]{}
	\fancyfoot[C]{}
	\fancyfoot[R]{\thepage}
	\renewcommand{\headrulewidth}{0.4pt}
	\renewcommand{\footrulewidth}{0pt}}

	\fancypagestyle{myfirstpage}{%Firstpage
	\fancyhf{}
	\fancyhead[L]{}
	\fancyhead[C]{}
	\fancyhead[R]{}
	\fancyfoot[L]{}
	\fancyfoot[C]{}
	\fancyfoot[R]{\thepage}
	\renewcommand{\headrulewidth}{0pt}
	\renewcommand{\footrulewidth}{0pt}}

	\pagestyle{myfency}

%================================================================================

%***************************
%*** Additional Command ****
%***************************

\DeclareMathOperator{\cl}{cl}
%================================================================================
%Document
\begin{document}
\thispagestyle{myfirstpage}
\begin{center}
	\Large{Functional Analysis HW8}
\end{center}
박성빈, 수학과, 20202120

\noindent \textbf{7.1}
\begin{proof}
For $1<p<\infty$, we get
\begin{equation}
    \abs{f+g}^p\leq \abs{f}\abs{f+g}^{p-1} + \abs{g}\abs{f+g}^{p-1}.
\end{equation}
for any measurable function $f,g$. Let's apply H\"older's inequality for $q=p/(p-1)$:
\begin{equation}
\begin{split}
    \norm{f+g}_p^p&=\int \abs{f+g}^p d\mu \\
    &\leq \int \abs{f}\abs{f+g}^{p-1}d\mu+\int\abs{g}\abs{f+g}^{p-1} d\mu\\
    &\leq \norm{f}_p\norm{\abs{f+g}^{p-1}}_q + \norm{g}_p\norm{\abs{f+g}^{p-1}}_q.
\end{split}
\end{equation}
In the last inequality, equality holds if and only if there exists $(\alpha_1,\beta_1)\neq (0,0)$, $(\alpha_2,\beta_2)\neq (0,0)$ such that $\alpha_1\abs{f}^p = \beta_1\abs{f+g}^p$ and $\alpha_2\abs{g}^p = \beta_2\abs{f+g}^p$ a.e.

Now, assume that for $h\in L^p$ such that $\norm{h}_p = 1$, and there exists $f,g$ in the unit ball of $L^p$ and $0<\lambda<1$ such that $h = \lambda f+(1-\lambda)g$. Since $\norm{h}_p\leq \lambda\norm{f}_p+(1-\lambda)\norm{g}_p$ with $\norm{f}_p,\norm{g}_p\leq 1$, $\norm{f}_p = \norm{g}_p = 1$.
For $q=(p-1)/p$,
\begin{equation}
\begin{split}
    1 &= \norm{\lambda f+(1-\lambda)g}^p_p \\
    &\leq  \lambda \norm{f}_p\norm{\abs{\lambda f+(1-\lambda)g}^{p-1}}_q+(1-\lambda)\norm{g}_p\norm{\abs{\lambda f+(1-\lambda)g}^{p-1}}_q \\
    &= \norm{\abs{\lambda f+(1-\lambda)g}^{p-1}}_q = 1.
\end{split}
\end{equation}
It means that the inequality should be equality and $\alpha\abs{f}^p = \beta\abs{g}^p$ for some $(\alpha,\beta)\neq (0,0)$ a.e. Since $\alpha \norm{f}_p^p = \beta\norm{g}_p^p$, $\alpha=\beta$ and $\abs{f}=\abs{g}$ a.e. Now, there exists $\theta_f,\theta_g\in [0,2\pi)$ such that $e^{i\theta_f}\abs{f}=f$ and $e^{i\theta_g}\abs{g} = g$. Finally, we get $h = (\lambda e^{i\theta_f}+(1-\lambda)e^{i\theta_g})\abs{f}$ and $\abs{\lambda e^{i\theta_f}+(1-\lambda)e^{i\theta_g}}\leq 1$ with equality if and only if $e^{i\theta_f}=e^{i\theta_g}$. To satisfy $\norm{h}_p=\norm{f}_p$, $e^{i\theta_f}$ should be same as $e^{i\theta_g}$ on the set $\{x:\abs{f(x)}>0\}$ a.e. so $f=g=h$ a.e. Therefore, $h$ is a extreme points of ball $L^p(\mu)$. Since extreme point of ball $L^p(\mu)$ is contained in $\{f\in L^p(\mu):\norm{f}_p=1\}$, the set of extreme points is the set.
\end{proof}

\noindent \textbf{7.9}
\begin{proof}
Assume there exists $x_1,x_2\in K$ with $0<\lambda<1$ such that $\lambda x_1+(1-\lambda)x_2 = x$. Then $y=T(x) = \lambda T(x_1)+(1-\lambda) T(x_2)$. Since $y$ is an extreme point of $T(K)$, $T(x_1)=T(x_2) = y$. Since $x$ is an extreme point and $x_1,x_2\in T^{-1}(y)$, it means that $x_1=x_2=x$. Therefore, $x$ is an extreme point.
\end{proof}

\noindent \textbf{7.10}
\begin{proof}
Let $T$ is an isometry, and there exists $T_1,T_2\in \mathscr{B}(\mathscr{H})$ with $0<\lambda<1$ such that $T=\lambda T_1+(1-\lambda)T_2$. Then, for any $x$ such that $\norm{x} = 1$,
\begin{equation}
    1 = \norm{x}^2 = \norm{Tx}^2 = \lambda^2\norm{T_1x}^2+(1-\lambda)^2\norm{T_2 x}^2 + 2\lambda(1-\lambda)\textrm{Re}\langle T_1x, T_2x\rangle.
\end{equation}
Since $\abs{\langle T_1x, T_2x\rangle}\leq \norm{T_1x}\norm{T_2x}$, $1\leq (\lambda\norm{T_1x}+(1-\lambda)\norm{T_2x})^2$. As $\norm{T_1}, \norm{T_2}\leq 1$, $(\lambda\norm{T_1x}+(1-\lambda)\norm{T_2x}) = 1$ and $\norm{T_1x}=\norm{T_2x} = \textrm{Re}\langle T_1x, T_2x\rangle = 1$. Now, assume that there exists $(T-T_1)x\neq 0$. WLOG, assume $\norm{x} = 1$ by dividing its norm, then
\begin{equation}
    0\neq \norm{(T-T_1)x}^2 = (1-\lambda)^2\norm{(T_1-T_2)x}^2 = (1-\lambda)^2\left(\norm{T_1x}^2+\norm{T_2x}^2-2\textrm{Re}\langle T_1x, T_2x\rangle \right) = 0,
\end{equation}
which is contradiction. Therefore, $T_1=T_2=T$.

Now, assume $T^*$ is an isometry, and there exists $T_1,T_2\in \mathscr{B}(\mathscr{H})$ with $0<\lambda<1$ such that $T=\lambda T_1+(1-\lambda)T_2$. Then we get $T^* = \lambda T_1^*+(1-\lambda)T_2^*$ with $T^*,T_1^*,T_2^*\in \mathscr{B}(\mathscr{H})$. As we saw above, $T_1^*=T_2^* = T^*$, so $T_1=T_2=T$ and $T$ is an extreme point.
\end{proof}
%________________________________________________________________________
\end{document}

%================================================================================